\documentclass[14pt, a4paper]{article}
\usepackage{extsizes} % Enables ability to change font size
\usepackage{indentfirst}

% Font settings
\usepackage[utf8]{inputenc}
\usepackage[T1,T2A]{fontenc}
\usepackage[russian,english]{babel}
\usepackage{tempora}
\linespread{1.15}

% Indentation settings
\usepackage[a4paper, left=30mm, right=15mm, top=20mm, bottom=20mm]{geometry}

% Paragraph offset size
\setlength\parindent{1.25cm}

\begin{document}
\selectlanguage{russian}

\section*{Введение}
Babel - это JavaScript-транскомпилятор, используемый в основном для преобразования кода ECMAScript 
2015+ (ES6+) в более раннюю версию JavaScript. Инструмент применяется для того, чтобы иметь 
возможность использовать все современные возможности языка при разработке и одновременно с этим 
обеспечить совместимость написанного кода со старыми версиями браузеров, не поддерживающими эти новые 
возможности.

Особенностью Babel является его модульная архитектура - трансформации, применяемые к коду, 
описываются в отдельных плагинах, которые можно подключать по отдельности, либо в виде пресета 
(группы плагинов). Также существует возможность создавать свои собственные плагины, чем я и собираюсь
 заняться в рамках выпускной квалификационной работы.

Целью этой работы является создание плагина, позволяющего использовать собственные синтаксические 
конструкции в контексте JavaScript кода. Моя мотивация к выбору этой тематики заключается в желании 
получить более глубокое представление о транспиляции программного кода, а также об устройстве работы 
наиболее популярного из JavaScript транспайлеров - Babel. Кроме того, проделанная работа может 
послужить доказательством концепции (proof of concept), если возникнет желание предложить внесение 
вышеупомянутых собственных синтаксических конструкций в стандарт языка JavaScript.

Данная работа состоит из двух частей. В первой части представлен обзор

\end{document}