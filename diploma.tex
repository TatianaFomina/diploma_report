\documentclass[14pt, a4paper]{article}
\usepackage{extsizes} % Enables ability to change font size
\usepackage{indentfirst}

% Font settings
\usepackage[utf8]{inputenc}
\usepackage[T1,T2A]{fontenc}
\usepackage[russian,english]{babel}
\usepackage{tempora}
\linespread{1.15}

% Indentation settings
\usepackage[a4paper, left=30mm, right=15mm, top=20mm, bottom=20mm]{geometry}

% Paragraph offset size
\setlength\parindent{1.25cm}

% Configure links
\usepackage[spaces,hyphens]{url}
\usepackage[colorlinks,allcolors=blue]{hyperref}
\urlstyle{same}

% Configure colors
\usepackage{color}
\definecolor{lightgray}{rgb}{0.9647058823529412,0.9725490196078431,0.9803921568627451}
\definecolor{blue}{rgb}{0.0,0.3607843137254902,0.7843137254901961}
\definecolor{green}{rgb}{0.27058823529411763,0.49411764705882355,0.2901960784313726}
\definecolor{purple}{rgb}{0.65, 0.12, 0.82}

% Configure listings
\usepackage{listings}
\lstset{
  basicstyle=\fontsize{12}{13}\ttfamily,
  columns=fullflexible,
  breaklines=true,
  postbreak=\mbox{\textcolor{red}{$\hookrightarrow$}\space},
  escapeinside={\%*}{*)},
  backgroundcolor=\color{lightgray},
  numberstyle=\footnotesize,
  tabsize=2,
}

% Use style=linesWithNumbers to enable lines enumerating
\lstdefinestyle{linesWithNumbers}{
  numbers=left,
}

\lstdefinelanguage{JavaScript}{
  keywords={break, continue, delete, else, for, function, if, in,
  new, return, this, typeof, var, void, while, with, const, false, null, true, boolean, number, undefined,
  Array, Boolean, Date, Math, Number, String, Object},
  keywordstyle=\color{blue}\bfseries,
  commentstyle=\color{green}\ttfamily,
  stringstyle=\color{purple}\ttfamily,
  sensitive,
  morecomment=[s]{/*}{*/},
  morecomment=[l]//,
  morecomment=[s]{/**}{*/}, % JavaDoc style comments
  morestring=[b]',
  morestring=[b]"
  fontadjust=true,
  numbersep=10pt,
}[keywords, comments, strings]


\begin{document}
\selectlanguage{russian}

\section*{Введение}
Babel $-$ это JavaScript-транскомпилятор, используемый в основном для преобразования кода ECMAScript 
2015+ (ES6+) в более раннюю версию JavaScript. Инструмент применяется для того, чтобы иметь 
возможность использовать все современные возможности языка при разработке и одновременно с этим 
обеспечить совместимость написанного кода со старыми версиями браузеров, не поддерживающими эти новые 
возможности.

Особенностью Babel является его модульная архитектура $-$ трансформации, применяемые к коду, 
описываются в отдельных плагинах, которые можно подключать по отдельности, либо в виде пресета 
(группы плагинов). Также существует возможность создавать свои собственные плагины, чем я и собираюсь
 заняться в рамках выпускной квалификационной работы.

Целью этой работы является создание плагина, позволяющего использовать собственные синтаксические 
конструкции в контексте JavaScript кода. Моя мотивация к выбору этой тематики заключается в желании 
получить более глубокое представление о транспиляции программного кода, а также об устройстве работы 
наиболее популярного из JavaScript транспайлеров $-$ Babel. Кроме того, проделанная работа может 
послужить доказательством концепции (proof of concept), если возникнет желание предложить внесение 
вышеупомянутых собственных синтаксических конструкций в стандарт языка JavaScript.

Данная работа состоит из двух частей. В первой части представлен обзор

\pagebreak
\section{Базовые концепции}
\subsection{ECMAScript и JavaScript}
ECMAScript $-$ это скриптовый язык программирования общего назначения, стандартизированный международной 
организацией ECMA в спецификации ECMA-262 \cite{ecma-262}. Спецификация ECMA-262 содержит правила и рекомендации, 
которые должны соблюдаться языком программирования, чтобы он считался совместимым с ECMAScript.

ES $-$ сокращение от ECMAScript. Каждая версия языка ECMAScript именуется с помощью сокращения ES и 
номера версии соответственно. Первая версия языка (ES1)  была выпущена в 1997 году. Самое значимое 
обновление язык получил с выходом стандарта версии ES6 (он же ES2015), в котором был добавлен новый 
синтаксис для описания классов, а также поддержка стрелочных функций, констант,  переменных с ограниченной областью 
видимости и так далее.

JavaScript $-$ язык программирования, являющийся реализацией стандарта ECMA-262. Другими словами, 
JavaScript расширяет язык ECMAScript, привнося в него дополнительные возможности.

\subsection{Babel}
Согласно официальной документации \cite{documentation} Babel является JavaScript компилятором. Однако, 
использование термина компилятор здесь не совсем уместно. Обычно под компилятором понимается программа, 
преобразующая исходные тексты программ, написанные на языке программирования высокого уровня, 
непосредственно в машинные инструкции. В большинстве других источников Babel называют транспайлером. 
Транспайлер (или транскомпайлер) $-$ это программа, преобразующая исходный код на одном языке высокого 
уровня в код на другом языке высокого уровня. В случае с Babel $-$ производятся преобразования между 
различными версиями языка JavaScript.

Необходимость в транспилировании JavaScript заключается в желании обеспечить совместимость написанного 
кода с максимальным числом браузеров. Дело в том, что большинство браузеров используют свой отдельный 
JavaScript интерпретатор (в Chrome используется V8, в Firefox $-$ SpiderMonkey, а в Internet Explorer $-$ Chakra), 
и каждый из этих интерпретаторов поддерживает свое независимое подмножество возможностей ES6 (2015). Это 
означает, что код одного и того же приложения может работать для пользователей одних браузеров и не 
работать для пользователей других. Именно эту проблему решает транспиляция кода с помощью Babel. Она 
позволяет использовать современный стандарт JavaScript при разработке и одновременно с этим обеспечить 
широкую поддержку браузеров.

Помимо решения проблемы совместимости, транспайлеры играют важную роль в процессе принятия решений 
комитетом TC39, группой специалистов, ответственных за разработку стандарта ECMAScript. Например, 
Babel поддерживает все экспериментальные возможности языка (находящиеся на рассмотрении комитетом), 
для того чтобы собрать отзывы от реальных пользователей, которые в свою очередь могут повлиять на 
решение членов комитета.


\subsection{Аналоги Babel}
Среди аналогов Babel можно выделить 2 наиболее масштабных, на мой взгляд, проекта: 
\textit{Traceur} и \textit{JSTransform}.

\textit{Traceur} \cite{traceur} - JavaScript транспайлер, предшественник Babel, презентованный компанией Google в 2011 году. 
Ввиду того, что проект перестал поддерживаться разработчиками с 2016 года, Traceur имеет заметно 
меньшую поддержку новых возможностей JavaScript по сравнению с Babel. Также на ограниченную 
функциональность повлиял архитектурный подход, выбранный разработчиками Traceur. Все проводимые над 
кодом трансформации заданы жестко внутри самого транспайлера, что несколько усложняет процесс расширения 
функциональности инструмента. В отличие от монолитной архитектура Traceur, Babel использует систему 
плагинов, с помощью которых описываются трансформации кода. Плагины могут разрабатываться отдельно 
от самого транспайлера и подключаться по востребованию.

\textit{JSTransform} \cite{jstransform} $-$ утилита, разрабатываемая в компании Facebook с 2013 по 2015 год, и изначально используемая 
в процессе сборки проектов на ReactJS. В первую очередь JSTransform позиционировался как
инструмент именно для создания и последующего применения к исходному коду собственных синтаксических преобразований, 
нежели как готовый ES6-to-ES5 транспайлер. По этой причине проект по умолчанию включал лишь небольшой 
набор предопределенных трансформаций. C 2015 проект JSTransform не поддерживается разработчиками и 
не рекомендуется к использованию.

\subsection{Как работает Babel}
*AST*

*AST picture*

Три основных этапа работы Babel - это парсинг, трансформация и генерация. Далее представлено описание каждого из этих этапов.

\subsubsection*{Парсинг}
На этом этапе код, передаваемый в Babel, конвертируется в структуру, называемую \textit{абстрактным 
синтаксическим деревом} (abstract syntax tree, AST). В свою очередь этап парсинга подразделяется на 
две фазы: \textit{лексический анализ} и \textit{синтаксический анализ}.

В ходе лексического анализа текст исходного кода преобразуется в массив токенов. По этой причине фазу 
лексического анализа часто называют токенизацией. Здесь токен - это объект, описывающий любую значащую 
подпоследовательность символов исходного кода. Например, фрагмент кода

\lstinputlisting[language=JavaScript]{listings/fragment-for-tokenization.js}
будет преобразован в следующий массив токенов:

\lstinputlisting[language=JavaScript]{listings/tokens.js}
Свойства start и end описывают положение токена в строке, loc $-$ строку, в которой был найден токен. 
Поле type указывает на объект, который хранит набор свойств, описывающих каждый отдельный токен:
\lstinputlisting[language=JavaScript]{listings/token-type.js}

Далее следует так называемый синтаксический анализ, цель которого $-$ построение абстрактного 
синтаксического дерева на основе массива токенов, полученного на предыдущем этапе. В ходе этого 
процесса токены преобразуются в узлы дерева (Node), которые кроме уже описанных данных содержат также
информацию о типе узла. Примерами таких типов являются FunctionDeclaration, VariableDeclaration или 
ReturnStatement. Для рассмотренного выше фрагмента кода построенное парсером абстрактное синтаксическое 
дерево будет выглядеть следующим образом:

\lstinputlisting[language=JavaScript]{listings/ast-for-fragment.js}

\subsubsection*{Трансформация}
Как только абстрактное синтаксическое дерево построено, начинается этап трансформации. На этом этапе 
производится рекурсивный обход, во время которого добавляются, заменяются или удаляются узлы дерева.
Трансформации, проводимые над деревом, определяются списком подключенных плагинов и применяются в 
порядке их следования в этом списке.

*Visitors*

\subsubsection*{Генерация}
По окончании модификации, абстрактное синтаксическое дерево конверируется обратно в код. 
Происходит это следующим образом: производится обход дерева в глубину, в процессе которого строится строка,
представляющая модифицированный код.

\pagebreak

\begin{thebibliography}{9}
  \bibitem{ecma-262} Стандарт ECMA-262 [Электронный ресурс] $-$ Режим доступа: \linebreak
    \url{https://www.ecma-international.org/publications/standards/Ecma-262.htm}
  \bibitem{documentation} Документация Babel [Электронный ресурс] $-$ Режим доступа: \linebreak
    \url{https://babeljs.io/docs/en}
  \bibitem{traceur} Документация Traceur [Электронный ресурс] $-$ Режим доступа: \linebreak
    \url{https://github.com/google/traceur-compiler/wiki/Getting-Started}
  \bibitem{jstransform} Репозиторий JSTransform на сервисе GitHub [Электронный ресурс] $-$ Режим доступа:
    \url{https://github.com/facebookarchive/jstransform}
\end{thebibliography}
\end{document}